\begin{preface}
\addchaptertocentry{\prefacename}
La tecnologia è un mondo molto vasto e in continua evoluzione, diventato non solo determinante per lo sviluppo delle società, ma soprattutto invasivo nella quotidianità delle persone. La maggior parte delle azioni svolte nell'arco di una giornata includono l'uso di mezzi tra cui i cellulari, i computer o altri dispositivi con i quali si riesce a reperire o caricare informazioni in rete in breve tempo. Molte persone usano in maniera superficiale tali tecnologie e non sono consapevoli dei numerosi pericoli presenti in rete; non si è abituati ad immaginare che un'informazione digitale che viaggia nel misterioso cyber spazio abbia un ruolo determinante nella nostra vita a pari valore di qualsiasi 'contratto' cartaceo o verbale. 
Talvolta, si sente una frase del tipo: "Ma cosa devo proteggere? E da chi? Io non ho nulla da nascondere".

In un mondo in cui ogni cosa oramai è condivisa in rete da più dispositivi è necessario informarsi sulla sicurezza, ed incentivare lo sviluppo di sistemi di prevenzione affidabili in grado di anticipare una minaccia o, in certi casi, addirittura riparare il danno causato. Bisogna però tenere in considerazione che tali sistemi possano essere non solo costosi economicamente, ma anche in termini di risorse che occupano all'interno del dispositivo. È dunque essenziale durante lo sviluppo di un software tenere in considerazione il grado di soddisfazione del futuro cliente, il quale è influenzato da numerosi fattori come la facilità d'uso e velocità di risposta.

È in questo scenario che nasce \emph{SysBench}, un programma per effettuare un benchmark del sistema in seguito all'utilizzo del modulo di sicurezza \emph{Linux Kernel Runtime Guardian}. Essendo una new entry nel mercato, LKRG non ha ancora ricevuto molte recensioni o analisi delle prestazioni che permettano ad un utente di valutarne seriamente l'uso. Pertanto, viene proposto SysBench come software OpenSource per fornire dati concreti circa le prestazioni di questo modulo. 

\end{preface}