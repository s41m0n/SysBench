\addchap{Conclusioni}

\label{Chapter5}

In un mondo in cui proteggere le informazioni è diventato oramai un obiettivo di vitale importanza è necessario essere consapevoli dell'esistenza di software in grado di aumentare l'integrità e la sicurezza del proprio sistema. Bisogna però tener conto del trade-off sicurezza-tempo, due importanti fattori talvolta antagonisti, in quanto più si aggiungono layer protettivi al sistema, più aumenta il tempo di risposta. È necessario valutare attentamente le proprie necessità prima di installare qualsiasi materiale, decidendo quali sono e come si possono raggiungere gli obiettivi prestabiliti, valutando se si preferisce garantire un'ottima prontezza di risposta, o la sicurezza (e non è detto che non vi sia un modo per oltrepassare queste ulteriori difese), oppure se raggiungere una soluzione intermedia ricoprendo entrambe le qualità.

Pertanto, in seguito all'analisi sviluppata in questo elaborato, è vivamente consigliato l'uso del modulo \emph{Linux Kernel Runtime Guardian}, il quale non solo è sorprendentemente performante, ma garantisce anche l'integrità dei dati e rileva efficacemente molte minacce al sistema che comprometterebbero la sua sicurezza. Nelle macchine testate, nonostante fossero ambienti virtuali e non fisici, si sono ottenuti ottimi risultati che portano a valutarne positivamente l'uso non solo nella propria installazione locale, ma anche nei vari server utilizzati con obiettivi differenti. Infatti, si è valutato l'utilizzo di LKRG anche in calcolatori con installazioni di Linux meno recenti (ad esempio negli ATM), le quali per problemi di supporto non vengono aggiornate, non curandosi in questo modo di possibili nuovi vettori d'attacco che si sono sviluppati negli ultimi anni. Con un trascurabile calo delle performance, il modulo potrebbe offrire a questi sistemi un maggior livello di sicurezza, favorendone il continuo utilizzo.

Infine, un ulteriore aspetto ritenuto personalmente importante è l'utilizzo di software libero: rispetto ad uno definito "proprietario", la cui licenza non ne permette la modifica, lo studio, la ridistribuzione e la condivisione tenendo segreto il sorgente (l'utente è limitato al semplice utilizzo), un software pubblicato sotto i termini di una licenza di software libero concede lo studio, la modifica e la ridistribuzione del progetto stesso. In questo modo non solo l'utente è a conoscenza della struttura del software potendosi leggere il sorgente, ma può anche apportare dei miglioramenti e condividerli con il resto della community. È grazie a questa filosofia che il mondo Linux è attraente ed efficace, in quanto offre la possibilità di costruire gratuitamente e liberamente il proprio sistema senza avere vincoli nei confronti di nessuno. 

In conclusione, LKRG offre un servizio a dir poco eccezionale, facendo sì che anche l'utente meno esperto e senza disponibilità economiche possa difendersi da certi tipi di minacce in maniera semplice ed efficace subendo una diminuzione accettabile delle performance.

