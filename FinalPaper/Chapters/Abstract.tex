\begin{otherlanguage}{english}
\begin{abstract}
\addchaptertocentry{\abstractname}

Gli argomenti d'interesse di questo elaborato sono l'integrità e la sicurezza di un sistema operativo (principalmente Linux). In particolare, viene analizzato il recente software \emph{Linux Kernel Runtime Guardian}, concentrando l'attenzione sulle sue funzionalità ed i controlli effettuati al fine di mantenere il proprio ambiente sicuro.

Prima dell'analisi del sistema, vengono introdotti gli argomenti inerenti per comprendere lo scenario in discussione, quali la sua struttura e gli strumenti in possesso dall'utente per l'interazione. Successivamente vengono presentati sia il software preso in analisi, servendosi della documentazione fornita dall'autore, sia quello progettato ed utilizzato per la misurazione delle performance \emph{SysBench}, interamente sviluppato per calcolare la differenza in termini di tempi di esecuzione di determinate funzionalità in seguito al caricamento di LKRG. Infine, vengono proposti e commentati i risultati sperimentali ottenuti effettuando il test in vari sistemi differenti, al fine di valutarne positivamente o meno l'uso.

Il progetto è OpenSource e si può ottenere clonando il repository nel sito \url{https://bitbucket.org/SimoMagno/sysbench/src/master/}.

\end{abstract}
\end{otherlanguage}
